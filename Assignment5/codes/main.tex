\documentclass[journal,12pt,twocolumn]{IEEEtran}
\usepackage[shortlabels]{enumitem}
\usepackage{setspace}
\usepackage{gensymb}
\singlespacing
\usepackage[cmex10]{amsmath}
\usepackage{graphicx}

\usepackage{float}
\usepackage{amsthm}

\usepackage{mathrsfs}
\usepackage{txfonts}
\usepackage{stfloats}
\usepackage{bm}
\usepackage{cite}
\usepackage{cases}
\usepackage{subfig}

\usepackage{longtable}
\usepackage{multirow}

\usepackage{enumitem}
\usepackage{mathtools}
\usepackage{steinmetz}
\usepackage{tikz}
\usepackage{circuitikz}
\usepackage{verbatim}
\usepackage{tfrupee}
\usepackage[breaklinks=true]{hyperref}
\usepackage{graphicx}
\usepackage{tkz-euclide}
\newtheorem{definition}{Definition}[section]
\usetikzlibrary{calc,math}
\usepackage{listings}
    \usepackage{color}                                            %%
    \usepackage{array}                                            %%
    \usepackage{longtable}                                        %%
    \usepackage{calc}                                             %%
    \usepackage{multirow}                                         %%
    \usepackage{hhline}                                           %%
    \usepackage{ifthen}                                           %%
    \usepackage{lscape}     
\usepackage{multicol}
\usepackage{chngcntr}
\usetikzlibrary{automata, positioning}
\DeclareMathOperator*{\Res}{Res}

\renewcommand\thesection{\arabic{section}}
\renewcommand\thesubsection{\thesection.\arabic{subsection}}
\renewcommand\thesubsubsection{\thesubsection.\arabic{subsubsection}}

\renewcommand\thesectiondis{\arabic{section}}
\renewcommand\thesubsectiondis{\thesectiondis.\arabic{subsection}}
\renewcommand\thesubsubsectiondis{\thesubsectiondis.\arabic{subsubsection}}


\hyphenation{op-tical net-works semi-conduc-tor}
\def\inputGnumericTable{}  %%
\newtheorem{theorem}{Theorem}[section]
\newtheorem{defn}[theorem]{Definition}
\lstset{
%language=C,
frame=single, 
breaklines=true,
columns=fullflexible
}
\begin{document}

\newcommand{\BEQA}{\begin{eqnarray}}
\newcommand{\EEQA}{\end{eqnarray}}
\newcommand{\define}{\stackrel{\triangle}{=}}
\bibliographystyle{IEEEtran}
\raggedbottom
\setlength{\parindent}{0pt}
\providecommand{\mbf}{\mathbf}
\providecommand{\pr}[1]{\ensuremath{\Pr\left(#1\right)}}
\providecommand{\qfunc}[1]{\ensuremath{Q\left(#1\right)}}
\providecommand{\sbrak}[1]{\ensuremath{{}\left[#1\right]}}
\providecommand{\lsbrak}[1]{\ensuremath{{}\left[#1\right.}}
\providecommand{\rsbrak}[1]{\ensuremath{{}\left.#1\right]}}
\providecommand{\brak}[1]{\ensuremath{\left(#1\right)}}
\providecommand{\lbrak}[1]{\ensuremath{\left(#1\right.}}
\providecommand{\rbrak}[1]{\ensuremath{\left.#1\right)}}
\providecommand{\cbrak}[1]{\ensuremath{\left\{#1\right\}}}
\providecommand{\lcbrak}[1]{\ensuremath{\left\{#1\right.}}
\providecommand{\rcbrak}[1]{\ensuremath{\left.#1\right\}}}
\theoremstyle{remark}
\newtheorem{rem}{Remark}
\newcommand{\sgn}{\mathop{\mathrm{sgn}}}
\providecommand{\abs}[1]{\vert#1\vert}
\providecommand{\res}[1]{\Res\displaylimits_{#1}} 
\providecommand{\norm}[1]{\lVert#1\rVert}
%\providecommand{\norm}[1]{\lVert#1\rVert}
\providecommand{\mtx}[1]{\mathbf{#1}}
\providecommand{\mean}[1]{E[ #1 ]}
\providecommand{\fourier}{\overset{\mathcal{F}}{ \rightleftharpoons}}
%\providecommand{\hilbert}{\overset{\mathcal{H}}{ \rightleftharpoons}}
\providecommand{\system}{\overset{\mathcal{H}}{ \longleftrightarrow}}
	%\newcommand{\solution}[2]{\textbf{Solution:}{#1}}
\newcommand{\solution}{\noindent \textbf{Solution: }}
\newcommand{\cosec}{\,\text{cosec}\,}
\providecommand{\dec}[2]{\ensuremath{\overset{#1}{\underset{#2}{\gtrless}}}}
\newcommand{\myvec}[1]{\ensuremath{\begin{pmatrix}#1\end{pmatrix}}}
\newcommand{\mydet}[1]{\ensuremath{\begin{vmatrix}#1\end{vmatrix}}}
\numberwithin{equation}{subsection}
\makeatletter
\@addtoreset{figure}{problem}
\makeatother
\let\StandardTheFigure\thefigure
\let\vec\mathbf
\renewcommand{\thefigure}{\theproblem}
\def\putbox#1#2#3{\makebox[0in][l]{\makebox[#1][l]{}\raisebox{\baselineskip}[0in][0in]{\raisebox{#2}[0in][0in]{#3}}}}
     \def\rightbox#1{\makebox[0in][r]{#1}}
     \def\centbox#1{\makebox[0in]{#1}}
     \def\topbox#1{\raisebox{-\baselineskip}[0in][0in]{#1}}
     \def\midbox#1{\raisebox{-0.5\baselineskip}[0in][0in]{#1}}
\vspace{3cm}
\title{Assignment-5}
\author{Lanka Prasanna-CS20BTECH11029}
\maketitle
\newpage
\bigskip
\renewcommand{\thefigure}{\theenumi}
\renewcommand{\thetable}{\theenumi}

\text{Download all latex codes from:}
\begin{lstlisting}
https://github.com/PrasannaLanka/Assignment5/blob/main/Assignment5/codes/Assignment5.tex
\end{lstlisting}

\section*{PROBLEM UGC/MATH (2018 Dec-Math set-a ) Q.104}
Let $X_1,X_2, \cdots$ be i.i.d. $N(0,1)$ random variables. Let $S_{n}=X_{1}^2+X_{2}^2+\cdots+X_{n}^2.\forall n\geq 1. $Which of the following statements are correct?
\begin{enumerate}[(A)]
\setlength\itemsep{1em}
\item $\frac{S_{n}-n}{\sqrt{2}}\sim N(0,1)$ for all $n\geq 1$
\item For all $\epsilon > 0$,$\Pr{\brak{\abs{\frac{S_n}{n}-2}>\epsilon}}\to 0$ as $n \to \infty$
\item $\frac{S_{n}}{n} \to 1$ with probability 1
\item $\Pr({S_{n} \leq n+\sqrt{n}x}) \to \Pr({Y \leq x}) \forall x\in R$ ,where $Y \sim N(0,2)$
\end{enumerate}
 \section*{Solution}
 Given $X_1,X_2, \cdots$ follow normal distribution with mean 0 and variance 1.
\begin{align}
    f_{X_i}(x)=\frac{1}{\sqrt{2}\pi}e^{-\frac{x^2}{2}} ,i \in \cbrak{1,2,\cdots}
\end{align}
As $X_1,X_2,\cdots $ are i.i.d random variables therefore $X_{1}^2,X_
{2}^2,\cdots$ are also identical and independent.
We can write
\begin{align}
    E(X^2)=0 \label{eq:x2}
\end{align}
 \begin{enumerate}[(A)]
\item \begin{align}
    E\brak{\frac{S_{n}-n}{\sqrt{2}}}&=E\brak{\frac{\sum_{i}{(X_{i}^{2}-1)}}{\sqrt{2}}}\\
    &={\frac{\sum_{i}E{(X_{i}^{2}-1)}}{\sqrt{2}}}\label{eq:expectation}
\end{align}
From \eqref{eq:x2} we can write
\begin{align}
    E\brak{\frac{S_{n}-n}{\sqrt{2}}}=0
\end{align}
\begin{align}
    Var\brak{\frac{S_{n}-n}{\sqrt{2}}}&=Var\brak{\frac{\sum_{i}{(X_{i}^{2}-1)}}{\sqrt{2}}}\\
    &={\frac{\sum_{i}Var{(X_{i}^{2}-1)}}{\sqrt{2}}}
\end{align}
\begin{align}
    Var(X_{i}^2-1)&=\int_{-\infty}^{\infty}(X_{i}^2-1)^2 f_{X_{i}}(x)dx\\
    &=\int_{-\infty}^{\infty}(X_{i}^4+1-2X_{i}^{2}) f_{X_{i}}(x)dx\\
    &=2\label{eq:var}
\end{align}
\begin{align}
    Var\brak{\frac{S_{n}-n}{\sqrt{2}}}&=n\sqrt{2}    
\end{align}
\begin{theorem}[Strong law of large numbers]
\label{theorem1}
Let $X_1,X_2,\cdots $ be i.i.d random variables with same expectation($\mu$) and finite variance($\sigma^2$).Let $S_{n}=X_1+X_2+\cdots X_n$,Then as $n \to \infty$
\begin{align}
    \frac{S_n}{n} \xrightarrow{a.s}  \mu,
\end{align}
almost surely.
\end{theorem}
Hence from theorem \ref{theorem1} as $n \to \infty$
\begin{align}
    \brak{\frac{S_{n}-n}{\sqrt{2}}}\sim N(0,n\sqrt{2})
\end{align}
Hence \textbf{Option A is false.}



\item Given 
\begin{align}
    S_{n}=X_{1}^2+X_{2}^2+\cdots+X_{n}^2.\forall n\geq 1
\end{align}
Assume that For all $\epsilon > 0$,$\Pr{\brak{\abs{\frac{S_n}{n}-2}>\epsilon}}\to 0$ as $n \to \infty$ is true
\begin{theorem}[Weak law of large numbers]
\label{theorem}
Let $X_1,X_2,\cdots $ be i.i.d random variables with same expectation($\mu$) and finite variance($\sigma^2$).Let $S_{n}=X_1+X_2+\cdots X_n$,Then as $n \to \infty$
\begin{align}
    \frac{S_n}{n} \xrightarrow{i.p}  \mu,
\end{align}
in probability
\end{theorem}
Hence from theorem \ref{theorem} we can write 
\begin{align}
    \frac{S_n}{n} \xrightarrow{i.p} Var(X^2)
\end{align}
\begin{align}
    \implies \frac{S_n}{n} \xrightarrow{i.p} 1
\end{align}
in probability.
\begin{definition}[Convergence in probability]
A sequence of random variables $\cbrak{X_n}_{n\in N}$ is said to converge in probability (denoted by i.p) to X if
\begin{align}
    \lim_{n \to \infty} \Pr(\left| X_{n}-X\right|>\epsilon)=0 ,\forall \epsilon>0
\end{align}\label{in prob}
\end{definition}
From definition \ref{in prob} we can write,
\begin{align}
    \implies \Pr{\brak{\abs{\frac{S_n}{n}-1}>\epsilon}}\to 0,\forall \epsilon>0
\end{align}
But this is contradiction to our assumption.\\
Hence \textbf{Option B is false .}
 
 
 
 \item Given 
\begin{align}
    S_{n}=X_{1}^2+X_{2}^2+\cdots+X_{n}^2.\forall n\geq 1
\end{align}
Hence from theorem \ref{theorem1} we can write 
\begin{align}
    \frac{S_n}{n} \xrightarrow{i.p} Var(X)
\end{align}
\begin{align}
    \implies \frac{S_n}{n} \xrightarrow{a.s} 1
\end{align}
almost surely.
\begin{definition}[Almost sure convergence]
A sequence of random variables $\cbrak{X_n}_{n\in N}$ is said to converge almost surely or with probability 1 (denoted by a.s or w.p 1) to X if \label{with prob 1}
\begin{align}
    \Pr(\omega |X_n(\omega) \to X(\omega))=1
\end{align}
\end{definition}

From definition \ref{with prob 1} we can write,
\begin{align}
    \frac{S_{n}}{n} \xrightarrow{w.p.1} 1
\end{align}
with probability 1.\\
Hence \textbf{Option C is true}.



\item Consider,
\begin{align}
    E\brak{\frac{S_{n}-n}{\sqrt{n}}}=0
\end{align}
using \eqref{eq:x2} and \eqref{eq:expectation}.
\begin{align}
     Var\brak{\frac{S_{n}-n}{\sqrt{n}}}&=\frac{2n}{\sqrt{n}}\\
     &=2\sqrt{n}.
\end{align}
using \eqref{eq:var}.
\begin{theorem}[Central limit theorem]
\label{theorem3}
The Central limit theorem states that the distribution of the sample approximates a normal distribution as the sample size becomes larger,given that all the samples are equal in size,regardless of the distribution of the individual samples.
\end{theorem}
From theorem \ref{theorem3} we can write,
\begin{align}
    \brak{\frac{S_{n}-n}{\sqrt{n}}} \sim N(0,2 \sqrt{n})\label{eq:D}
\end{align}
\begin{align}
     \Pr{\brak{\frac{S_{n}-n}{\sqrt{n}} \leq x}}= \Pr{\brak{S_{n} \leq n+\sqrt{n}x}}
\end{align}
Hence using \eqref{eq:D}, \textbf{Option D is false.}
 
 \end{enumerate}
    
\end{document}